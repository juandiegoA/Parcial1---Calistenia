\documentclass{article}
\usepackage[utf8]{inputenc}
\usepackage[spanish]{babel}
\usepackage{listings}
\usepackage{graphicx}
\graphicspath{ {images/} }
\usepackage{cite}

\begin{document}

\begin{titlepage}
    \begin{center}
        \vspace*{1cm}
            
        \Huge
        \textbf{Parcial 1 - Calistenia}
            
        \vspace{0.5cm}
        \LARGE
        
            
        \vspace{1.5cm}
            
        \textbf{Juan Diego Arias Toro}
            
        \vfill
            
        \vspace{0.8cm}
            
        \Large
        Despartamento de Ingeniería Electrónica y Telecomunicaciones\\
        Universidad de Antioquia\\
        Medellín\\
        Marzo de 2021
            
    \end{center}
\end{titlepage}

\tableofcontents
\newpage
\section{Instrucciones}\label{intro}
Leer atentamente las instrucciones y ponerlas en práctica con extremo cuidado.

1.Utilizando únicamente la mano de su preferencia realice los siguientes pasos.

2.Iniciando en la posición A levantar la hoja de papel y apartarla a un lado de la superficie plana, horizontal en la que se encuentra realizando el ejercicio.

3.Con el dedo pulgar, índice, medio (también llamado corazón) y anular levantar las tarjetas, con el dedo pulgar, en uno de los filos laterales, y los dedos índice, medio y anular, tomar el otro dorso lateral.

4.Mientras tengan las tarjetas sujetadas como se indica en el paso 3, con el dedo meñique deslizar la hoja de nuevo hasta donde estaba  en la posición inicial.

5.Levantar el dedo meñique de la hoja de papel.

6.Pasa lentamente el dedo índice por los dorsos de las tarjetas hasta que se encuentre una separación, y  lentamente hacerla más grande.

*De no encontrar dicha separación repita el paso hasta que las tarjetas tomen la forma deseada
*Si desea puede emplear las uñas para hacer la separacion más facil

7.En la separación encontrada en el paso anterior introduzca su dedo indice sin dejar de hacer una ligera presión en los laterales con el dedo medio y pulgar.

8.Formar un triángulo con los filos superiores de las tarjetas.

9.Descender lentamente el triángulo hecho en el paso anterior hacia la hoja de papel y colocar los filos inferiores lentamente, mientras que con el dedo índice se sostiene los filos superiores para no perder la forma del triángulo de tal manera que cada una impida que la otra caiga sobre la hoja de papel.

10.En caso de que el triángulo no se mantenga estable y se caiga repita los pasos del 7 al 9 hasta conseguir una estructura estable.

\end{document}
